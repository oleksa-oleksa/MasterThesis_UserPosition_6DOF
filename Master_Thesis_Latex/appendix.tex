% !TeX spellcheck = de_DE
%\appendix
\clearpage
\pagenumbering{Roman}
\chapter*{Glossar}
\addcontentsline{toc}{chapter}{\protect\numberline{Glossar}}%
\section*{AJAX}
\label{sec:appendix:ajax}
AJAX (asynchrones Javascript und XML) ist der allgemeine Name für Technologien, mit denen asynchrone Anforderungen (ohne erneutes Laden von Seiten) an den Server gestellt und Daten ausgetauscht werden können. Da die Client- und Serverteile der Webanwendung in verschiedenen Programmiersprachen geschrieben sind, müssen zum Austausch von Informationen die Datenstrukturen (z. B. Listen und Wörterbücher), in denen sie gespeichert sind, in das JSON-Format konvertiert werden.

\section*{Daemon}
\label{sec:appendix:daemon}
Ein Unix-Daemon ist ein Programm, das "im Hintergrund" ausgeführt wird, ohne dass die Steuerung über ein Terminal erforderlich ist, und dem Benutzer die Möglichkeit bietet, andere Prozesse "im Vordergrund" auszuführen. Der Dämon kann entweder von einem anderen Prozess gestartet werden, z. B. von einem der Systemstartskripte, ohne auf ein Steuerterminal zuzugreifen, oder vom Benutzer von einem beliebigen Terminal aus. In diesem Fall "entführt" der Dämon das Terminal jedoch nicht, während es ausgeführt wird.

\section*{HOST}
\label{sec:appendix:host}
Host - Dies  der Name der IP-Adresse für den Webserver, auf den zugegriffen wird. Dies ist normalerweise der Teil der URL, der unmittelbar auf den Doppelpunkt und zwei Schrägstriche folgt.

\section*{HTTP}
\label{sec:appendix:http}
HTTP steht für HyperText Transfer Protocol, Hypertext Transfer Protocol". HTTP ist ein weit verbreitetes Datenübertragungsprotokoll, das ursprünglich für die Übertragung von Hypertextdokumenten vorgesehen war (Dokumente, die möglicherweise Links enthalten, mit denen Sie den Übergang zu anderen Dokumenten organisieren können). Die Basis dieses Protokolls ist eine Anforderung von einem Client (Browser) an einen Server und eine Serverantwort an einen Client.

\section*{JSON}
\label{sec:appendix:json}
JSON (JavaScript Object Notation) ist ein universelles Format für den Datenaustausch zwischen einem Client und einem Server. Es ist eine einfache Zeichenfolge, die in jeder Programmiersprache verwendet werden kann.

\section*{PORT} 
\label{sec:appendix:port}
Dies ist ein optionaler Teil der URL, der die Portnummer angibt, die der Zielwebserver abhört. Die Standardportnummer für HTTP-Server ist 80, einige Konfigurationen sind jedoch so eingerichtet, dass sie eine alternative Portnummer verwenden. In diesem Fall muss diese Nummer in der URL angegeben werden. Die Portnummer wird direkt mit einem Doppelpunkt, der unmittelbar auf den Servernamen oder die Adresse folgt, eingegeben.

\section*{URI}
\label{sec:appendix:uri}
Uniform Resource Identifier ist ein Pfad zu einer bestimmten Ressource (z.B.  einem Dokument), für die eine Operation ausgeführt werden muss (z. B. bei Verwendung der GET-Methode bedeutet dies das Abrufen einer Ressource). Einige Anforderungen beziehen sich möglicherweise nicht auf eine Ressource und in diesem Fall kann der Startzeile anstelle des URI ein Sternchen (Symbol "*") hinzugefügt werden.


 