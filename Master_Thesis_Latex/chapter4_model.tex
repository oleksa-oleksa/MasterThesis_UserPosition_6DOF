% !TeX spellcheck = en
\chapter{Data and Model}
\label{sec:design}

\section{6-DoF Dataset}
\label{sec:design:dataset}
This section describes how the dataset was obtained, analyzed and presents the visualization of user's head position and rotation. The real 6-DoF dataset must be used as training data from which the model can learn the spacial and time dependences. This step is crucial for a high accuracy prediction and almost all machine learning approaches requires not only row data collection but also data exploration and data preprocessing steps to be done before training begins.

\subsection{Data collection from HMD}
\label{sec:design:dataset:HL}
The low-cost IMU is sometimes unstable in frame rate during collecting data. In the experiments, we set the frame rate of IMU with 50 Hz which means that data will be collected per 0.02 second. Unfortunately, sometimes IMU could occur delay, and the time gap of two samples may be reduced to 0.01 second or increased to 0.05 second. To deal with above situation,

\subsection{Data Exploration}
\label{sec:design:dataset:explor}
!! Data analysis AVG linear velocity position, plots 

\subsection{Data preprocessing}
\label{sec:design:dataset:preprocessing}


\section{Neural Network}
\label{sec:design:nn}

\subsection{Network architecture}
\label{sec:design:nn:architecture}

\subsection{Network input}
\label{sec:design:nn:input}


\section{Training methods}
\label{sec:design:train}
