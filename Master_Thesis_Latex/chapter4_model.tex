% !TeX spellcheck = en
\chapter{Data and Model}
\label{sec:design}
This chapter presents the steps of development and implementation of the proposed approach. The Unity Application for HoloLens was deployed on HMD and a raw data with measures and dimension columns was obtained. This data than was analyzed and preprocessed to ensure that the captured data can be used in corresponding machine learning models. The model architecture was implemented and experimentally improved during training and evaluation steps. 

\section{6-DoF Dataset}
\label{sec:design:dataset}
This section describes how the dataset was obtained, analyzed and presents the visualization of user's head position and rotation. The real 6-DoF dataset must be used as training data from which the model can learn the spacial and time dependences. This step is crucial for a high accuracy prediction and almost all machine learning approaches requires not only row data collection but also data exploration and data preprocessing steps to be done before training begins.

\subsection{Data collection from HMD}
\label{sec:design:dataset:HL}
In this master thesis HoloLens 2, the second iteration of Microsoft's head-mounted mixed reality device, was used for data collection. The user position and orientation were obtained with Unity application developed for this purpose. Main Camera in Unity is automatically configured to track head movements. More details about Unity application can be found in section \ref{sec:imp:programming:unity}.\\
Using the Main Camera, a user position $(x, y, z)$ and orientation in quaternion $(qx, qy, qz, qw)$ were logged in a $csv$-file. Also additional parameters were recorded from the Main Camera in order to add more information during training processes. Thus the rotation as Euler angles in degrees and world-space speed of the camera in units per second and were added into dataset.\\
Even the high-cost HMD, like used in this research HoloLens 2, is sometimes unstable in frame rate during collecting data. In the Unity Application, we set the frame rate of IMU with 50 Hz which means that data will be collected per 0.02 second. Unfortunately, sometimes IMU could occur delay, and the time gap of two samples may be reduced to 0.01 second or increased to 0.05 second. To deal with above situation,

\footnote{https://docs.microsoft.com/en-us/windows/mixed-reality/develop/advanced-concepts/hologram-stability}

\subsection{Data Exploration}
\label{sec:design:dataset:explor}
!! Data analysis AVG linear velocity position, plots 

\subsection{Data preprocessing}
\label{sec:design:dataset:preprocessing}


\section{Neural Network}
\label{sec:design:nn}

\subsection{Network architecture}
\label{sec:design:nn:architecture}

\subsection{Network input}
\label{sec:design:nn:input}


\section{Training methods}
\label{sec:design:train}
