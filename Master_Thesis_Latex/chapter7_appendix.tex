% !TeX spellcheck = en_EN
%\appendix
\clearpage
\pagenumbering{Roman}
\chapter*{Glossary}
\addcontentsline{toc}{chapter}{\protect\numberline{Glossary}}%
\section*{AJAX}
\label{sec:appendix:ajax}
AJAX (asynchrones Javascript und XML) ist der allgemeine Name für Technologien, mit denen asynchrone Anforderungen (ohne erneutes Laden von Seiten) an den Server gestellt und Daten ausgetauscht werden können. Da die Client- und Serverteile der Webanwendung in verschiedenen Programmiersprachen geschrieben sind, müssen zum Austausch von Informationen die Datenstrukturen (z. B. Listen und Wörterbücher), in denen sie gespeichert sind, in das JSON-Format konvertiert werden.


 