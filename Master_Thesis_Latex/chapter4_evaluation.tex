% !TeX spellcheck = en

\chapter{Evaluation}
\label{sec:eval}

\section{Baseline model}
\label{sec:eval:baseline}

\section{Goal of evaluation}
\label{sec:eval:goal}

The paper of \textit{Chung et al., 2014}  also provides an interesting comparison and evaluation of the performance of recurrent units LSTM and GRU on sequence modeling. Authors mentioned the ability of LSTM to keep the existing memory via the introduced gates and thus to detect an important feature from an input sequence at early stage, to easily carry this information (the existence of the feature) over a long distance, hence, capturing potential long-distance dependencies \cite{empirical_evaluation}. The GRU takes linear sum between the existing state and the newly computed state similar to the LSTM but does not have any mechanism to control the degree to which its state is exposed, but exposes the whole state each time \cite{empirical_evaluation}. \textit{Chung et al., 2014} emphasize the fact that any important feature, decided by either the forget gate of the LSTM unit or the update gate of the GRU, will not be overwritten but be maintained as it is \cite{empirical_evaluation}. LSTM unit controls the amount of the new memory content and does not have any separate control of the amount of information flowing from the previous time step. The GRU differs and controls the information flow from the previous activation when computing the new and does not independently control the amount of the candidate activation being added via update gate \cite{empirical_evaluation}.

\section{Evaluation metrics}
\label{sec:eval:metrics}

\section{Experiments}
\label{sec:eval:experiments}

\subsection{First experiments}
\label{sec:eval:experiments:early}

\subsubsection{Datasets}
\label{sec:eval:experiments:early:ds}
As already stated in section \ref{sec:design:dataset:preprocessing}

\subsubsection{Batch size}
\label{sec:eval:experiments:early:batch}
A high impact on the performance e.g. the prediction accuracy has a batch size used in LSTM or GRU Model. The batch-size helps to learn the common patterns as important features by providing a fixed number of samples at one time. So that the model thus can distinguish the common features by looking at all the introduced samples of the batch. In most cases, an optimal batch size is set to 64. When this batch size was initially used with LSTM model, it gave significant high MSE, RMSE, train and validation errors. Based on the performance observation during experiments with LSTM parameters, batch size fine-tuning was done. The experiments done by \textit{Aykut et al} in their works \cite{delay_compensation_360} and \cite{telepresence} proved that appropriate batch size can be found in range $2^{9}$ - $2^{11}$ (512 - 2048). Notice that a power of 2 is used as a batch size. The overall idea is to fit a batch of samples entirely in the the CPU/GPU. Since, all the CPU/GPU comes with a storage capacity in power of two, it is advised to keep a batch size a power of two. Using a number different from a power of 2 could lead to poor performance.

\subsubsection{Learning rate}
\label{sec:eval:experiments:early:lr}

\subsection{Prediction with LSTM}
\label{sec:eval:experiments:lstm}
!!!
During preprocessing step Euler angles (yaw, pitch, roll) were calculated from quaternions and these parameters are used for visualization purposes. Although the interpolated $csv$-file contains additional Euler angles columns, only described in section \ref{sec:design:dataset:HL} parameters were used for training and prediction.

\subsection{Prediction with GRU}
\label{sec:eval:experiments:gru}

\subsection{Prediction with Bidirectional GRU}
\label{sec:eval:experiments:bi-gru}