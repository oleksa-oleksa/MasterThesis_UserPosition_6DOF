% !TeX spellcheck = de_DE
\chapter{Zusammenfassung}
\label{sec:results}
Die Fähigkeit eines IoT-Geräts, OTA-Updates zu empfangen, hat für die Behebung von Sicherheitslücken eine entscheidende Bedeutung. OTA-Updates wirken sofort, um die Implementierung robust zu halten und den Datenschutz zu gewährleisten. OTA-Dienste müssen schnell, sicher und einfach zu verwenden sein. Das Übertragen dieser Art von Updates ist jedoch nicht einfach, da es eine Reihe von Kompetenzen umfasst, z.B. das Verwalten verschiedener Versionen der Firmware, damit ein Fehler im Update das Gerät nicht „sperrt“, oder ein dringendes Update zum richtigen Zeitpunkt durchgeführt wird.

Die Isolation und der Integritätsschutz der Verarbeitungsumgebung können auf unterschiedliche Weise erfolgen. In dieser Arbeit wurden die offenen Standards untersucht, die allgemeine Bausteine für sichere Firmware-Updates auf eingeschränkten IoT-Geräten bereitstellen. 