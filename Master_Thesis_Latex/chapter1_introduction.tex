% !TeX spellcheck = en
\chapter{Introduction}
\label{sec:intro}
This thesis is focusing on designing and evaluation of the new approach for the predicting human head position within a 6-dimensional degree of freedom (6-DoF) for Extended Reality (XR) applications. The recent improvements in computer graphics, connectivity and the computational power of mobile devices simplified the progress in Virtual Reality (VR) technology. The way users can interact with their devices changed dramatically. With new technologies of VR environment user becomes the main driving force in deciding which portion of media content is being displayed to them at any time of interaction with VR Applications \cite{new_challenge}. Until recently the high-quality experiences with modern Augmented Reality (AR) and VR systems were not widely presented in home usage and were mainly used in research labs or commercial setups. The hardware for displaying the VR environment was once extremely expensive but recent years became more broadly accessible and the 6-DoF VR headset designed for the end-user were released\footnote{https://medium.com/@DAQRI/motion-to-photon-latency-in-mobile-ar-and-vr-99f82c480926}. It is possible now to experience virtual reality scenes and watch new type of volumetric media at home and the market interest for development VR and AR applications expected to be huge next years.


\section{Problem statement}
\label{sec:intro:problem}
The existing on this moment virtual environments can be divided into two main groups. Depending on position of the user and their ability to move inside the VR environment all the 3-DoF and 6-DoF. 

\section{Motivation for the research}
\label{sec:intro:motivation}
The following chapter introduces the preprocessing pipeline, architecture, and evaluation process used to develop ahead motion prediction algorithm.


\section{Structure of the thesis}
\label{sec:intro:structure}
The organization of this thesis is as follows. The literature review chapter introduces the concepts of XR technologies and principles of motion prediction algorithms. It follows an overview of previous research. 

\textbf{Chapter 1} - Introduction.\\
The following chapter introduces the preprocessing pipeline, architecture, and evaluation process used to develop ahead motion prediction algorithm.




