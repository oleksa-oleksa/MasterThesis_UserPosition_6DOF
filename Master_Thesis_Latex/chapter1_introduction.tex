% !TeX spellcheck = en
\chapter{Introduction}
\label{sec:intro}
This thesis is focusing on designing and evaluation of the approach for the prediction of human head position in a 6-dimensional degree of freedom (6-DoF) of Extended Reality (XR) applications for a given look-ahead time (LAT) in order to reduce the Motion-to-Photon (M2P) latency of the network and computational delays. At the beginning of the work the existing head motion prediction methods were analysed, and their similarities differences will be taken into account when a proposed Recurrent Neural Network-based predictor will be developed. Main goal is the systematic analysis of the potential of recurrent neural networks for head motion prediction. The proposed approach was evaluated on a real head motion dataset collected from Microsoft HoloLens. Based on a discussion of the obtained results, suggestions for future work are provided.

\section{Problem statement}
\label{sec:intro:problem}
The correct and fast head movement prediction is a key to provide a smooth and comfortable user experience in VR environment during head-mounted display (HDM) usage. The recent improvements in computer graphics, connectivity and the computational power of mobile devices simplified the progress in Virtual Reality (VR) technology. The way users can interact with their devices changed dramatically. With new technologies of VR environment user becomes the main driving force in deciding which portion of media content is being displayed to them at any time of interaction with VR Applications \cite{new_challenge}. Until recently the high-quality experiences with modern Augmented Reality (AR) and VR systems were not widely presented in home usage and were mainly used in research labs or commercial setups. The hardware for displaying the VR environment was once extremely expensive but recent years became more broadly accessible and the 6-DoF VR headset designed for the end-user were released\footnote{https://medium.com/@DAQRI/motion-to-photon-latency-in-mobile-ar-and-vr-99f82c480926}. It is possible now to experience virtual reality scenes and watch new type of volumetric media at home and the market interest for development VR and AR applications expected to be huge next years.\\
In fact, the existing on this moment virtual environments can be divided into two main groups depending on position of the user and their ability to move inside the VR environment. The user motion and prediction within a 3-DoF environment has been intensely researched for years. Extending such approaches to a 6-DoF environment is not straightforward, due to the change of the user's viewing point from inward to outward and additional three degrees of freedom \cite{6-dof_metrics}.\\
Although all mentioned above improvements, rendering of volumetric content remains very demanding task for existing devices. Thus the improvement of a performance of existing methods, design and implementation of new approaches specially for the 6-DoF environment could be a promising research topic.

\section{Motivation for the research}
\label{sec:intro:motivation}

Research efforts to reduce the computational load are being already wide attempted. However, these approaches designed for the client side. Recently presented technique of the rendering on a cloud server makes possible to decrease the computational load on the client device by offloading of the task to a server infrastructure and than by sending the rendered 2D content instead of volumetric data \cite{serhan_cloud_streaming}. The calculated 2D view must correspond the current position and orientation of a user. However, cloud-based streaming approach adds network latency and processing delays due uploading to a server the user position, rendering a new 2D picture from the 3D data and sending it back to a device. Thus, a rendered 2D image can appear even later on a display than with usage of local rendering system.\\
The promising research topic is a reducing the Motion-to-Photon (M2P) latency by predicting the future user position and orientation for a look-ahead time (LAT) and sending the corresponding rendered view to a client. The LAT in this approach must be equal or larger to the M2P latency of the network including round-trip time (RTT) and time need for calculation and rendering of a future picture at remote server.

\section{Structure of the thesis}
\label{sec:intro:structure}
The organization of this thesis is as follows. The thesis starts from introduction and problem statement, followed by theoretical background related to the research topic. Literature review chapter introduces different approaches and  technologies of motion prediction algorithms. The chapters \ref and \ref{sec:eval} show the implementation ff the presented models and evaluation of the results that were obtained during experiments. Last, the discussion regarding method limitations and suggestions for the future work are done.

\textbf{Chapter 1} - Introduction.\\
The current chapter shortly introduces a state of development on scientific field achieved at a time of master thesis creation in the context on XR applications. The necessity of timely action to improve the situation with increasing computational and network latency is shown in problem statement section \ref{sec:intro:problem}. Due to the breadth of the research topic, the section \ref{sec:intro:motivation} focuses and motivates the research topic. 

\textbf{Chapter 2} - Background.\\
The next chapter includes a review of the area being researched. It starts with a short introduction of the concept of MR applications and presents a 6-DoF environment. The presence and influence of a computational and network latency is covered. In section \ref{sec:theorie:head_pred} the challenges faced in predicting of the viewer's position are discussed. Last section contains an overview of previous research in the field of prediction of user's head position and orientation and places a master thesis's topic in the context of the existing literature. 

\textbf{Chapter 3} - Implementation.\\
Chapter describes practical implementation of the approach. The dataset including data collection from head mounted display (HDM) and data understanding and preprocessing are described in section \ref{sec:impl:dataset}. Model Inputs, Model architecture and the development steps are covered in section \ref{sec:impl:model}. The implementation of Unity Application, training and evaluation loop with PyTorch and hyperparameter search described in the section \ref{sec:impl:model:dev:programming}.


\textbf{Chapter 5} - Evaluation.\\
A Baseline model, used for comparing the obtained results and tuning the hyperparameters is described. The goal of evaluation and metrics used in this research are covered. The conducted experiments with a data obtained from HMD for each analysed RNN Model can be found in section \ref{sec:eval:experiments}.

\textbf{Chapter 5} - Conclusion.\\
The last chapter presents a discussion about the limitation of proposed method and provides a conclusion about the received results including suggestions for potential types of future research.