% !TeX spellcheck = en_EN
\chapter{Introduction}
\label{sec:intro}
This thesis is focusing on designing and evaluation of the new approach for the predicting human head motion within a 6-dimensional degree of freedom (6-DoF) for Extended Reality (XR) applications. The recent improvements in computer graphics, connectivity, wearable technology and rising the computational power of mobile devices available for regular users and last progress on VR technology changed dramatically the way user can interact with new devices and watch new immerse media. With new technologies of VR environment user becomes the main driving force in deciding which portion of media content is being displayed to them at any time of interaction with VR Applications \cite{new_challenge}. Until recently the high-quality experiences with modern AR and VR systems were not widely presented in home usage and were mainly used in research labs or commercial setups. The things are changed and recent years the several the 6-DoF the virtual reality headset developed by Oculus VR 


\section{Problem statement}
\label{sec:intro:problem}
The existing on this moment virtual environments can be divided into two main groups. Depending on position of the user and their ability to move inside the VR environment all the 3-DoF and 6-DoF. 

\section{Motivation for the research}
\label{sec:intro:motivation}
The following chapter introduces the preprocessing pipeline, architecture, and evaluation process used to develop ahead motion prediction algorithm.


\section{Structure of the thesis}
\label{sec:intro:structure}
The organization of this thesis is as follows. The literature review chapter introduces the concepts of XR technologies and principles of motion prediction algorithms. It follows an overview of previous research. 

\textbf{Chapter 1} - Introduction.\\
The following chapter introduces the preprocessing pipeline, architecture, and evaluation process used to develop ahead motion prediction algorithm.




