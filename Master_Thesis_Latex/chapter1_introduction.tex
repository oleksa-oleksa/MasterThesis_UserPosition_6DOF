% !TeX spellcheck = de
\chapter{Einleitung}
\label{sec:intro}
Aufgrund des schnellen Tempos, mit dem sich das IoT entwickelt und die ständig steigen Zahl der IoT Geräten, besteht es ein zunehmender Bedarf, Software-Updates für Sicherheitsaspekten, Fehlerkorrekturen und Software-Erweiterungen zu unterstützen. Für die kleinste IoT-Geräte ist es zu aufwändig, die Updates mittels angeschlossenen USB-Leitung hochzuladen und ausführen. Deshalb wird eine Softwareaktualisierung über eine Funkschnittstelle (typischerweise WLAN oder Mobilfunknetz) durchgeführt, somit kann sogar die geschlossenen zwischen Wanden Geräte erreicht werden. Ein englischen Begriff "Over-the-Air-Update" bedeutet Aktualisierung über Luft". Diese Technologie hat bereits eine ganze Entwicklung hinter sich. Zu Beginn der 2000er Jahre, in der Zeit vor dem Telefonieren, was es gewohnt SMS-Nachrichten mit Bildern an das Telefon zu senden oder mit SMS einen kostenpflichtigen Bildschirmschoner zu bestellen. Zu diesem Zweck musste ein spezieller digitaler Code, der aus dem Katalog ausgewählt wurde, an eine kurze Nummer gesendet werden. Siemens-Telefone gehörten damals mit ihrem „Siemens OTA“ zu den Pionieren, der als Standard für die Übertragung nicht nur von Bildern und Bildschirmschonern, sondern auch von WAP-Einstellungen, Kontakten, Kalenderereignissen usw. konzipiert wurde.

OTA Update ist praktisch der Industriestandard für die Aktualisierung von Software auf einem mobilen Gerät. Der reine Over-the-Air (OTA) Updates sind in ihrer Funktionalität zu begrenzt. Es war nur die manuelle Installation von Konfigurationsprofilen zulässig, keine neuen Profile automatisch gepullt werden können, die Geräte und seine Sicherheitseinstellungen konnten nicht automatisch aktualisiert werden, keine Befehle gab es zu Verfügung. Heutzutage OTA ist mit der Verwendung vom größeren Geräteverwaltungssystems MDM (Device Management System) verstärkt, das einen Standort des IoT-Geräts und Status des Updates überwacht und somit es ermöglicht unter anderem das Aktualisieren von Software oder Konfiguration des IoT-Geräts. Auf der Geräteseite kommuniziert OTA-/DM-Klient mit dem Server. Viele MDM-Anbieter verwenden OTA zum Booten von MDM (Installieren eines Konfigurationsprofils mit MDM-Nutzdaten auf einem Gerät) und später MDM für den Rest der Geräteverwaltung.

\section{Ziel der Arbeit}
\label{sec:intro:ziel} 
OTA-Updates (Over-the-Air) bieten viele Vorteile für IoT-Geräte (Internet of Things). Sie ermöglichen das Remote-Patchen von Fehlern oder Sicherheitslücken, anstatt dass Servicetechniker kostenpflichtig oder unerfahrene Benutzer unsicher die Updates persönlich durchführen müssen. 

In dieser Arbeit wird die Anforderungen an einen sicheren und zuverlässigen OTA-Aktualisierungsmechanismus \cite{website:SecureFirmware} für die eingebettete Software und Hardware auf dem IoT-Gerät untersucht und gleichzeitig die wichtigsten Open-Source-Protokolle behandelt. Es wird über eine Reihe von Sicherheitsproblemen diskutiert, die das OTA-System lösen muss, um anwendbar zu sein: das Gerät muss dem Server vertrauen, der Server muss dem Gerät vertrauen, das Gerät darf kein Image ausführen, dem es nicht vertraut \cite{website:Attestation}, \cite{website:Principles}, \cite{website:TrustLite}. Häufig ist es auch erforderlich, das Image selbst geheim zu halten, wie bei seinem übertragen und wenn es bereits auf dem Gerät ist. 

